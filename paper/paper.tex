\documentclass{acm_proc_article-sp}

\begin{document}

\title{Crowdsourcing}
\subtitle{An overview of the scientific state of the art}

\numberofauthors{4} 
\author{
% 1st. author
\alignauthor Stefan Derkits\\
       \email{stefan@derkits.at}
% 2nd. author
\alignauthor Manuel Mertl\\
       \email{manuel.mertl@gmail.com}
% 3rd. author
\alignauthor Felix Winter\\
       \email{ixos01@gmail.com}
\and  % use '\and' if you need 'another row' of author names
% 4th. author
\alignauthor Christian Wagner\\
       \email{christian.wagner86@gmx.at}
}

\maketitle
\begin{abstract}
This paper tries to give an overview of the current state of the art of Crowdsourcing. Crowdsourcing has proven to be an efficient approach for problems like relevance evaluation \cite{alonso2008crowdsourcing}.
\end{abstract}

\category{H.5.3}{Group \& Organization Interfaces}{Collaborative computing}


\terms{Theory}

\keywords{Crowd sourcing, Distributed problem solving} 

\section{Introduction}

\section{A section}


\subsection{A subsection}


\section{Conclusions}

\bibliographystyle{abbrv}
\bibliography{paper}

\end{document}
